\documentclass[10pt,a4paper,twoside]{article}


\usepackage[utf8]{inputenc}
\usepackage[T1]{fontenc}
\usepackage{array, tabularx, multirow, booktabs} % Amélioration de
	% l'environnement tabular, calculer automatiquement la taille des colonnes et
	% permettre la fusion de ligne et facilite la production de tables telles
	% qu’elles apparaissent (devraient apparaître) dans les livres et journaux
	% scientifiques publiés
\usepackage{chemist, eurosym, numprint, xfrac}
\usepackage{color, graphicx}
\usepackage{coolstr, ifthen, xspace, xstring}
\usepackage{csquotes, enumitem, soulutf8}
% à passer en dernier
\usepackage[french]{isodate, babel}
\usepackage{hyperref}

\author{%
Frédérique  \nom{Cohet}\\%
Maître de Conférences de l’Université Panthéon-Assas Paris \II
}
\title{%
Droit des Baux Civils\\
Master 2 << Droit immobilier et de la construction >>}
\date{2020}

\hypersetup{%
%	pdfinfo={%
%		Title={Droit de l'Urbanisme et de l'Aménagement}%
%		, Subject={}%Marché \@ReferenceMarche}%
%		, Author={Samuel Déom}%
%		%, Keywords={}%\@ReferenceMarche}%
%	}%
	, colorlinks = true% colore, plutot qu'encadre, les liens hypertexte
	, linkcolor = black% colore les liens internes en noir
	, urlcolor = black% colore les liens externes en noir
	, breaklinks = true% autorise les liens à être étendus sur plusieurs lignes
}

% Ecrire du texte juridique
\makeatletter
\newcommand*{\assPlen}{\@ifstar{\mbox{ass. plén.}\xspace}{assemblée
	plénière\xspace}}
\newcommand*{\chMixte}{\@ifstar{ch. mixte\xspace}{chambre mixte\xspace}}
\newcommand*{\civUn}{\@ifstar{civ. 1\iere{}\xspace}{première chambre
	civile\xspace}}
\newcommand*{\civDeux}{\@ifstar{civ. 2\ieme{}\xspace}{deuxième chambre
	civile\xspace}}
\newcommand*{\civTrois}{\@ifstar{civ. 3\ieme{}\xspace}{troisième chambre
	civile\xspace}}
\newcommand*{\CourDeCas}{\@ifstar{Cass.\xspace}{Cour de Cassation\xspace}}
\newcommand*{\CA}{\@ifstar{CA\xspace}{Cour d'Appel\xspace}}
\newcommand*{\CAA}{\@ifstar{CAA\xspace}{Cour Administrative d'Appel\xspace}}
\newcommand*{\CE}{\@ifstar{CE\xspace}{Conseil d'État\xspace}}
\newcommand*{\CJCE}{\@ifstar{CJCE\xspace}{Cour de justice des Communautés
	européennes\xspace}}
\newcommand*{\CJUE}{\@ifstar{CJUE\xspace}{Cour de justice de l'Union
	européenne\xspace}}
\newcommand*{\jurisCourDeCas}{\@ifstar{\jurisCourDeCasStared%
	}{\jurisCourDeCasNotStared}}
\newcommand*{\jurisCourDeCasStared}[3][]{\CourDeCas* #2*,
	\printdate{#3}\ifthenelse{\equal{#1}{}}{}{, \no#1}} % Numéro de chambre, date
	% et en option no de pourvoi
\newcommand*{\jurisCourDeCasNotStared}[3][]{\CourDeCas #2,
	\printdate{#3}\ifthenelse{\equal{#1}{}}{}{, \no#1}} % Numéro de chambre, date
	% et en option no de pourvoi
\newcommand*{\jurisCE}[2][]{\CE*, \printdate{#2}\ifthenelse{\equal{#1}{}}{}{,
	\no#1}} % Date et en option no de pourvoi
\newcommand*{\jurisCA}[2]{\CA* #1, \printdate{#2}} % Ville, date
\newcommand*{\jurisCAA}[3][]{\CAA* #2,
	\printdate{#3}\ifthenelse{\equal{#1}{}}{}{, \no#1}} % Ville, date, et en
	% option no de pourvoi

\newcommand*{\refArticle}[2]{\mbox{#1.\,#2}}

\newcommand*{\articleCodifie}{\@ifstar{\articleCodifieStared%
	}{\articleCodifieNotStared}}
\newcommand*{\articleCodifieStared}[2]{\mbox{art.~\refArticle{#1}{#2}}}
\newcommand*{\articleCodifieNotStared}[2]{article~\refArticle{#1}{#2}}
\makeatother
\newcommand*{\articlesCodifies}[2]{articles~\refArticle{#1}{#2}}
\newcommand*{\ArticleCodifie}[2]{Article~\refArticle{#1}{#2}}
\newcommand*{\ArticlesCodifies}[2]{Articles~\refArticle{#1}{#2}}
\newcommand*{\articlesCodifiesEtSuivants}[2]{\articlesCodifies{#1}{#2} et
	suivants}
\newcommand*{\articleDu}[3][]{\ifthenelse{\equal{#1}{}}{article~\mbox{#2}%
	}{\articleCodifie{#1}{#2}} du #3}
\newcommand*{\articlesDu}[3][]{\ifthenelse{\equal{#1}{}}{articles~\mbox{#2}%
	}{\articlesCodifies{#1}{#2}} du #3}
\newcommand*{\articlesDuEtSuivants}[3][]{\ifthenelse{\equal{#1}{}%
	}{articles~\mbox{#2}}{\articlesCodifies{#1}{#2}} et suivants du #3}
\newcommand*{\ArticleDu}[3][]{\ifthenelse{\equal{#1}{}}{Article~\mbox{#2}%
	}{\ArticleCodifie{#1}{#2}} du #3}
\newcommand*{\ArticlesDu}[3][]{\ifthenelse{\equal{#1}{}}{Articles~\mbox{#2}%
	}{\ArticlesCodifies{#1}{#2}} du #3}
\newcommand*{\ArticlesDuEtSuivants}[3][]{\ifthenelse{\equal{#1}{}%
	}{Articles~\mbox{#2}}{\ArticlesCodifies{#1}{#2}} et suivants du #3}

% Les codes et leur abbréviation (tiré de )
\makeatletter
\newcommand*{\casf}{\@ifstar{CASF}{Code de l'action sociale et des
	familles}\xspace}
\newcommand*{\cassur}{\@ifstar{\mbox{C.~assur.}\xspace}{Code des
	assurances\xspace}}
\newcommand*{\cch}{\@ifstar{CCH}{Code de la construction et de
	l'habitation}\xspace}
\newcommand*{\cciv}{\@ifstar{\mbox{C.~civ.}}{Code civil}\xspace}
\newcommand*{\ccom}{\@ifstar{\mbox{C.~com.}}{Code du commerce}\xspace}
\newcommand*{\cconsom}{\@ifstar{\mbox{C.~consom.}}{Code de la
	consommation}\xspace}
\newcommand*{\ccp}{\@ifstar{CCP}{Code de la commande publique}\xspace}
\newcommand*{\cgi}{\@ifstar{CGI}{Code général des impôts}\xspace}
\newcommand*{\cpc}{\@ifstar{CPC}{Code de procédure civile}\xspace}
\newcommand*{\cpen}{\@ifstar{\mbox{C.~pén.}}{Code pénal}\xspace}
\newcommand*{\cpi}{\@ifstar{CPI}{Code de la propriété intellectuelle}\xspace}
\newcommand*{\cpp}{\@ifstar{CPP}{Code de procédure pénale}\xspace}
\newcommand*{\crur}{\@ifstar{\mbox{C.~rur.}}{Code rural et de la pêche
	maritime}\xspace}
\newcommand*{\curb}{\@ifstar{\mbox{C.~urb.}}{Code de l'urbanisme}\xspace}
\makeatother

\newcommand*{\nom}[1]{\textsc{#1}}
\newcommand*{\pourcent}[1]{\nombre{#1}~\%}
\newcommand*{\montant}[1]{\mbox{\nombre{#1}~\euro}}
\newcommand*{\milliard}[1]{\mbox{\nombre{#1}~M~\euro}}
\newcommand*{\surface}[1]{\nombre{#1}~\metreCarre}

% Sans arguments
\makeatletter
\newcommand*{\bofip}{\@ifstar{BOFiP}{bulletin officiel des finances
	publiques}\xspace}
\makeatother
\newcommand*{\cad}{c'est-à-dire\xspace}
\makeatletter
\newcommand*{\colloc}{\@ifstar{collectivité locale}{collectivités
	locales}\xspace}
\makeatother
\newcommand*{\dmto}{droits de mutation à titre onéreux\xspace}
\newcommand*{\etc}{\emph{etc}.\xspace}
\newcommand*{\II}{\textsc{ii}\xspace}
\newcommand*{\III}{\textsc{iii}\xspace}
\newcommand*{\IR}{impôt sur le revenu\xspace}
\newcommand*{\JP}{jurisprudence\xspace}
\newcommand*{\metreCarre}{m\up{2}}
\newcommand*{\NP}{nu-propriétaire\xspace}
\newcommand*{\OP}{ordre publique\xspace}
\newcommand*{\PV}{procès-verbal\xspace}
\newcommand*{\tva}{TVA\xspace}



% Environnement
\newenvironment*{exemple}{Par exemple :\newline\itshape}{}
\newenvironment*{conseil}{{\bfseries Conseil :}\smallbreak\itshape}{}
\newenvironment*{casPratique}[2][]%
{\stepcounter{casPratiqueCtr}{\bfseries Cas pratique \no
	\arabic{casPratiqueCtr}}
	\ifthenelse{\equal{#1}{}}{}{\itshape #1}
	{#2}}%
{}
\newenvironment*{focus}[1][]{\medskip \textbf{#1} \newline \itshape}{}

\begin{document}

\maketitle

\section*{Généralités}

	\paragraph{Un contrat nommé} Le contrat de bail est un contrat nommé, c'est un
		contrat qui est spécialement régit par le \cciv aux articles 1709 à 1778.
	Mais c'est un contrat qui peut en plus être soumis à une réglementation
		complémentaire, qui relève du \enquote{droit spécial des baux} --- qui pour
		l'essentiel est une réglementation d'\OP.

	\paragraph{Le contrat de bail} L'\articleDu{1709}{\cciv} dispose qu'un contrat
		de bail est un \enquote{contrat par lequel l’une des parties s’oblige à
		faire jouir l’autre d’une chose pendant un certain temps, et moyennant un
		certain prix que celle ci s’oblige à lui payer.}
	Cette définition permet la qualififcation d'un contrat en contrat de bail et
		met en avant ses principales caractéristiques :
		\begin{enumerate}
			\item fourniture de la jouissance d’un bien pour une durée
				déterminée\footnote{C'est-à-dire qui ne peut pas être perpétuelle} ;
			\item mise à disposition à usage exclusif\footnote{Précisé par la \JP
				\jurisCourDeCas[04-19736]{\civTrois*}{11/01/2006}} ;
			\item paiement d’un loyer\footnote{À défaut : commodat ou prêt à usage.
				L'absence de loyer suffisament consistant, qui ne correspond pas à la
				valeur locative, peut aussi conduire à l'annulation du contrat ---
				\jurisCA{Paris}{28/01/2016}, AJDI 2016 p. 425. ; \jurisCourDeCas[%
				13-26291]{\civTrois*}{24/03/2015}}.
		\end{enumerate}

	Ce contrat est à distinguer des conventions d’hébergement et de fourniture de
		prestation de services contrat hotelier ou contrat d’entreprise. Il y a
		beaucoups de \JP à ce sujet, par exemple à propos de la location de longue
		durée d'une chambre d'hôtel.

	Il est également à distinguer des conventions génératrices de droits réels ---
		en particulier : bail réel solidaire, bail emphythéotique et bail à
			construction --- car le contrat de bail est un droit \textbf{personnel}.

	\paragraph{Le différentes catégories de baux civils} Les grandes catégories de
		baux qui relèvent de la dénomination des \enquote{baux civils} sont :
		\begin{itemize}
			\item les baux de locaux loués nus à usage d’habitation ou mixte
				(évolution législative 1948, 1982, 1986, 1989, 2014, 2018) ;
			\item les baux de locaux loués meublés à usage d’habitation ;
			\item les baux de locaux à usage professionnel ;
			\item les baux ruraux ;
			\item les baux du secteur libre.
		\end{itemize}

	\paragraph{L'application exclusive aux baux du secteur libre} Ces baux civils
		sont des contrats qui peuvent être exclusivement soumis aux dispositions du
		\cciv\ --- non seulement le droit des contrats mais également le
		\enquote{droit spécial des baux} --- lorsqu'ils font partis de ce que l'on
		appelle le \enquote{secteur libre.}

		Le baux du secteur libre sont pour l'essentiels ceux listés ci-après :
		\begin{itemize}
			\item baux de résidences secondaires ;
			\item baux de locaux accessoires ;
			\item logements de fonction\footnote{Il pourrait paraitre surprenant de
				qualifier ce type de contrat de bail. Ces contrats mettant à
				disposition un logement dans l'exercice d'une fonction ne sont pas
				tous qualifiés de bail (cf. \hl{On en reparlera tout à l'heure}). Un
				logement de fonction peut être mis à disposition en lien étroit avec
				l'exercice d'une fonction et ne pas présenter les caractéristiques du
				contrat de bail et en conséquence ne pas pouvoir être qualifié de
				\enquote{bail}. Dans ce cas ce sera une convention d'occupation
				précaire. Tout dépendra du point de savoir si les deux parties ont
				entendus faire de cette convention de mise à disposition d'un logement
				un contrat autonome par rapport au contrat qui gouverne l'exercice des
				fonctions du bénéficiaire du logement. \jurisCourDeCas[16-15743]{
				\civTrois*}{22/06/2017}} ;
			\item baux consentis a des personnes pour leur activité
				economique\footnote{Attention : il ne s'agit pas ici d'une activité
				commerciale ou d'une activité libérale} ;
			\item baux de terrains nus a usage non agricole.
		\end{itemize}

		Pour tous ces baux, il n'existe pas de réglementation impérative.

	\paragraph{L'application supplétive aux baux du secteur réglementé} Bon nombre
		de baux, qui sont soumis à des reglementations particulières, se voient
		appliquer les règles des baux civils, de manière supplétive.

\section{Le droit commun des baux civils}

	Le droit commun des baux doit être appréhendé en lien avec le droit commun
		des contrats, puisque les baux sont des contrats.

	\paragraph{Effet de la réforme des obligations de 2016} L'ordonnance \no 2016-
		131 du \printdate{10/02/2016}, réformée par la loi \no 2018-287 du
		\printdate{20/04/2018} de ratification, concerne donc le droit des baux et
		peut influencer le régime de droit commun des baux civils.
	Il fut donc concerné par la polémique sur la question de savoir comment ce
		droit s'applique dans le temps, notamment en ce qui concerne les baux
		\enquote{loi de 89}.

	La question se posait notamment de savoir si ce nouveau droit s'appliquait aux
		contrats en cours, et plus particulièrement aux effets qui se produisent
		postérieurement à l'entrée de la loi nouvelle mais qui trouvent leur cause
		dans un contrat ou une situation qui est antérieure à celle-ci.

	Après un avis de la \CourDeCas du \printdate{16/02/2015}, \no 14-70011,
		plaidant pour application immédiate de la loi nouvelle aux contrats en cours
		s’agissant des effets légaux du contrat, appuyé par un arrêt indiquant que
		la jurisprudence lit le droit à la lumière de \enquote{l’évolution du droit
		des obligations résultant de l’ordonnance \no 2016-131 du
		\printdate{10/02/2016}\footnote{\jurisCourDeCas[15-20411]{\chMixte}{
		24/02/2017}}}, la loi de ratification du \printdate{20/04/2018} a modifié
		l’article 9 de l'ordonnance du \printdate{10/02/2016} en prévoyant que
		\enquote{les contrats conclus avant cette date demeurent soumis à la loi
		ancienne \emph{y compris pour leurs effets légaux et pour les dispositions
		d'ordre public}.}

	Cela a été logiquement confirmé par la \civTrois de la \CourDeCas dans son
		arrêt \no 18-20854 du \printdate{19/12/2019}.

	La loi nouvelle s'applique donc aux contrats conclus à compter du
		\printdate{1/10/2016} (ou du \printdate{1/10/2018}) sauf en ce qui concerne
		les règles interprétatives.

	\paragraph{Contrat d’adhesion} Il s'impose un \hl{clin d'oeil} au contrat
		d'adhésion qui, bien qu'existant préalablement la réforme de 2016, a été
		consacré par celle-ci en lui donnant une définition précise.

	L'\articleDu{1110 al. 2}{\cciv} stipule que \enquote{le contrat d'adhésion
		est celui qui comporte un ensemble de clauses non négociables, déterminées
		à l'avance par l'une des parties.}

	Un contrat de bail peut souvent répondre à cette définition dès lors que le
		preneur n'est pas en mesure de discuter des clauses essentielles du
		contrat.

	Cette catégorisation en un contrat d'adhésion n'a d'interêt que dans le cas
		où il n'y a pas de réglementation d'ordre publique. En effet, les
		conséquences à cette qualification sont inapplicable aux dispositions
		d’ordre publique.

	Cette distinction ne présente donc un interêt que pour les baux du secteur
		libre.
	Dans ce cas, en application de l'article 1171 nouveau, qui prévoit que
		\enquote{dans un contrat d'adhésion, \emph{toute clause non négociable,
		déterminée à l'avance par l'une des parties}, qui crée un déséquilibre
		significatif entre les droits et obligations des parties au contrat est
		réputée non écrite}, s'il y a appréciation d'un déséquilibre significatif
		qui ne porte ni sur l'objet principal du contrat ni sur l'adéquation du
		prix à la prestation, il sera possible d'écarter les clauses incriminées.

	On notera que le champ d'application de cet article 1171 est limité car il
		ne s'appliquera que là où il n'y a pas d'autre réglmentation pouvant gérer
		les rapports entre les parties.

	En effet, l'article 1105 nouveau rappelle que le spécial déroge au général,
		et permet de considérer que dans les rapports entre un professionel
		bailleur et un particulier locataire c'est le droit de la consommation qui
		s'appliquera, et donc les dispositions propres à celui-ci concernant la
		lésion\footnote{\ArticleDu[L]{212-1}{\cconsom} issu de l’ordonnance du
		\printdate{14/03/2016}}, ou qu'entre professionels c'est le code de la
		concurrence qui trouvera à s'appliquer\footnote{Le plus simple est de faire
		déclarer à chaque partie qu'elle a été en mesure de négocier les termes du
		contrat et permettre ainsi d'éviter tout débat quant à la qualification du
		contrat en contra d'adhésion soumis à l'article 1171 nouveau du Code civil.

		Un exemple d’application possible du texte est donné par
			\jurisCourDeCas[14-25523]{\civTrois*}{17/12/2015} : \enquote{la clause qui
			fait peser sur le locataire la quasi totalité des dépenses incombant
			normalement au bailleur et dispense sans contrepartie le bailleur de toute
			participation aux charges qui lui incombent normalement en sa qualité de
			propriétaire, a pour effet de créer, au détriment du consommateur, un
			déséquilibre significatif entre les droits et obligations des parties au
			contrat.}

		Depuis la loi Alur \no 2014-366 du \printdate{24/03/2014}, qui a réformé la
			loi de 89, la solution est applicable pour les dispositions non prévues
			dans les contrats types au contenu impératif, sous réserve que le contrat
			soit entre deux particuliers.}.

	\begin{quote}
		\enquote{les contrats, qu'ils aient ou non une dénomination propre, sont
			soumis à des règles générales, qui sont l'objet du présent sous titre.

		Les règles particulières à certains contrats sont établies dans les
			dispositions propres à chacun d'eux.

		Les règles générales s'appliquent sous réserve de ces règles
			particulières.}
	\end{quote}

	\subsection{Formation du contrat de bail}

		Elle est soumise au droit commun des contrats\footnote{
			\ArticlesDuEtSuivants{1101}{\cciv}} mais doit également respecter des
			exigences qui propres au contrat de bail.

		Ces exigences propres concernent tout autant les parties que l’objet de
			leurs obligations\footnote{Ce que l'on appelle depuis l'ordonnance : le
			contenu du contrat}, qui doit être licite et certain.

		\subsubsection{Les parties au contrat}

			\paragraph{Consentement} Comme pour tout contrat, il est indispensable que
				chaque partie aie pu exprimer un consentement éclairé.

				\subparagraph{Objet} Le consentement doit se rencontrer relativement à
					la chose louée : quel bien est mis à la disposition du locataire par
					le bailleur ? Il faut que cet objet soit déterminé, sa durée doit être
					précisée\footnote{Il s'agit ici de définir cette durée, qui peut être
					\enquote{indeterminée.} Exception pour le bail judiciaire en cas de
					divorce : la durée est fixée par le juge.} ainsi que la contre-partie
					du locataire \cad le loyer.

				\subparagraph{Caractéristiques} Le consentement doit exister et être
					exempt de vices, notamment :
					\begin{itemize}
						\item erreur sur la personne --- puisque le contrat est conclu
							\emph{intuitus personae},
						\item réticence dolosive comme un situation du locataire falsifiée
							--- faux bulletins de paie,
						\item le vice de violence\footnote{Le nouvel article 1143 du \cciv
							précise : \newline
							\enquote{Il y a également violence lorsqu’une partie abusant de
							l’état de dépendance dans lequel se trouve son cocontractant
							\emph{à son égard} obtient de lui un engagement qu’il n’aurait pas
							souscrit en l’absence d’une telle contrainte et en tire un
							avantage manifestement excessif}}, en particulier lorsque l'un des
							cocontractant est en situation de dépendance de l'autre par
							ailleurs (ex. travail).
					\end{itemize}

				\subparagraph{Forme du consentement} Le consentement peut être exprimé
					librement, le contrat de bail est un contrat par nature
					consensuel\footnote{\ArticleDu{1714}{\cciv}} : l'écrit n'est pas
					indispensable à la validité mais il est exigé à titre de preuve.

				Il existe cependant une petite exception concernant les baux de plus de
					12 ans, dont l'\articleDu{710-1}{\cciv} exige la publication à la
					publicité foncière --- ce qui suppose l'existence d'un acte
					authentique. A défaut il y a inopposabilité aux tiers pour la période
					excédant les 12 ans\footnote{\jurisCourDeCas[05-10794]{\civTrois*%
					}{7/03/2007}}.

				\begin{focus}{Preuve du bail verbal}

					\textbf{Existence du consentement} La preuve de l'existence même du
						contrat de bail se fera différemment selon que le contrat a reçu un
						début d'execution ou non.
					Si le bail a reçu un début d’execution, on pourra apporter la preuve
						de son existence par tout moyen.
					Cela peut être par le biais de quittance, par témoignage, par échange
						de courriel, \etc
					La \JP est relativement exigente s'agissant de cette preuve, et il a
						été jugé que le simple fait que des virement aient été fait ne
						suffisez pas à caractériser l'existence d'un contrat de
						bail\footnote{\jurisCA{Paris}{11/05/2007} : un local mis à
						disposition par une grand-mère à son petit-fils, en l'abscence de
						quittance de loyer et considérant que les transferts financiers
						pouvaient être expliqués par d'autres raisons.}.
					Si le bail n'a pas reçu de début d'execution, la preuve par serment
						déféré à celui qui nie le bail\footnote{\ArticleDu{1715}{\cciv}} est
						le seul moyen admis par le \cciv pour établir l'existence du bail.
					La preuve du commencement d’exécution peut, par contre, être établie
						par tous moyens.

					\textbf{Montant du loyer} Lorsque le bail a reçu un début
						d’exécution la preuve du montant du loyer se fait uniquement par
						quittances, serment du bailleur ou estimation par expert.
					Ces règles sont d’\OP, et le juge ne peut fixer le loyer en fonction
						d’autres éléments.

					\textbf{Durée} En l'absence d'écrit, il est considéré que le bail est
						à durée indéterminée\footnote{Avec l'inconvénient déjà évoqué de la
						possibilité pour chacune des parties de mettre fin au contrat à tout
						moment, sous réserve d'un préavis raisonnable.} --- en l’absence de
						règles d’ordre public contraires\footnote{Par exemple : un bail
						soumis à la loi de 89 sera d'une durée de trois ans ou de six ans,
						suivant la qualité du bailleur.}.

					\medskip L'abscence d'écrit présente un inconvénient majeur en ce
						sens qu'elle ne permet pas de prévoir des clauses d'aménagements des
						obligations réciproques des parties.
					Par exemple, s'il n'y a pas de bail écrit, il ne sera pas possible
						d'envisager de révision du loyer --- peu importe que le bail soit
						soumis à une réglementation d'\OP ou non.
				\end{focus}

			\paragraph{Capacité}

				S'agissant de la capacité, on retrouve bien évidemment le droit commun :
					toute partie doit être capable de conclure un contrat, et donc capacité d'exercice et capacité de jouissance.

				Cette question de la capacité présente des particularités tant pour le
					bailleur que pour le preneur.

				\subparagraph{Capacité du bailleur}

					Par nature, le bail est un acte d’administration\footnote{Mais
						l'\articleDu{504}{\cciv} encadre les baux consentis par le tuteur},
						mais ce principe souffre d'exception et un bail sera considéré comme
						un acte de disposition si sa durée supérieure à 9 ans ou s'il s'agit
						d'un bail rural ou commercial.
					En effet, dans ces deux dernières hypothèses, on a considéré que les
						obligations qui incombent au bailleurs sont tellement lourdes --- il
						doit maintenir le locataire en jouissance tant que celui remplie ses
						obligations.

				\subparagraph{Capacité du preneur}

					Du coté du preneur, c'est également un acte d’administration --- sans difficulté particulière.

					On relève toutefois certaines incapacités de jouissance lorsque le
						preneur est dans certaines situations --- en vue de protéger le
						bailleur --- par exemple :
						\begin{itemize}
							\item l'\articleDu{508}{\cciv} interdit au tuteur de devenir
								locataire des biens de son pupille --- sauf à obtenir une autorisation prélable du juge ou du conseil de famille ;
							\item l'\articleDu[L]{116-4}{\casf} frappe d'une incapacité de
								prendre à bail le personnel des établissements qui recoivent des
								personnes agées ou d'un trouble psychatrique les biens des ses
								personnes.
						\end{itemize}

			\paragraph{Pouvoir}

				\subparagraph{Pouvoir du bailleur} Le bailleur doit avoir le pouvoir de
					conclure le contrat de bail, ce qui suppose au minimum qu'il ait la
					jouissance du bien objet du contrat de bail : soit qu'il en soit le
					propriétaire, soit qu'il en ait l'usufruit, soit parce qu'il s'est vu
					reconnaitre un droit de jouissance avec la faculté de sous-location.

					\begin{enumerate}
						\item Le \textbf{bailleur non propriétaire} n'est autorisé à
							conclure un contrat de bail qui pourra être opposé au propriétaire
							que dans la mesure cela lui est autorisé.
						L'\articleDu{1717}{\cciv} pose le principe de la faculté de sous
							location, sous réserve que cette faculté n'ai pas été interdite
							--- soit par la  réglementation, soit par la convention entre les
							parties.

						\smallbreak Lorsque la sous-location est interdite, soit par la loi
							soit par la convention entre les parties, et qu'elle a quand même
							lieu, cela permet au bailleur d'agir en résiliation, car la sous-
							location constitue alors une faute de la part du preneur.
						La question du sort des loyers perçu par le preneur dans ce cas et
							plus largement le sort des fruits a été reposée assez largement
							ces dernières années avec l'essort des plateformes Internet de
							location saisonnière.
						Il a été ainsi jugé que le bailleur propriétaire qui avait découvert
							que son locataire avait sous-loué via Airbnb pouvait réclamer la
							restitution des sous-loyers\footnote{\jurisCourDeCas[18-
							20727]{\civTrois}{12/09/2019}}.

						\smallbreak Par ailleurs, la location de la chose d’autrui constitue
							une infraction pénale.
						L'\articleDu{313-6-1}{\cpen} prévoit en effet que :
							\begin{quote}
								\enquote{
									Le fait de mettre à disposition d'un tiers, en vue qu'il y
									établisse son habitation moyennant le versement d'une
									contribution ou la fourniture de tout avantage en nature,
									un bien immobilier appartenant à autrui, sans être en mesure
									de justifier de l'autorisation du propriétaire ou de celle du
									titulaire du droit d'usage de ce bien, est puni d'un an
									d'emprisonnement et de \nombre{15000} euros d'amende.
								}
							\end{quote}

						\item Lorsque le droit de propriété est démembré, le \NP ne peut
							revetir la qualité de bailleur, puisqu'il ne dispose plus de la
							jouissance du bien et ne peut donc la conférer à qui que ce soit, et seul l'usufruitier peut donner le bien à bail.
						L'\articleDu{595}{\cciv} pose expressement le principe de la liberté
							pour le \textbf{bailleur usufruitier} de donner à bail.

						\smallbreak Il peut cependant arriver que les particularités d'un
							contrat de bail imposent l'intervention du \NP, en particulier en
							ce qui concerne les baux ruraux et les baux commerciaux.
							\begin{enumerate}
								\item Les baux ruraux et commerciaux nécessitent l'accord
									préalable du \NP\ --- ou à défaut, dans le cas où le refus du
									propriétaire ne serait pas justifié, par autorisation
									judiciaire\footnote{\ArticleDu{595 al. 4}{\cciv}} --- à peine
									de nullité.

								L'objet est ici encore une fois de protéger le \NP des
									conséquences de la protection particulière dont bénéficient les locataires de ces types de baux et notamment la quasi impossibilité pour le \NP de mettre fin à ce contrat de bail.

								\item Pour les autres baux, la question de la liberté pour
									l'usufruitier de donner à bail dépend de la durée du contrat envisagé :
								\begin{itemize}
									\item Lorsque la durée du bail est de plus de \textbf{9 ans}
										l'autorisation du \NP est requise pour que le bail lui soit opposable au moment de la reconstitution de la pleine propriété.
									A défaut, l'\articleDu{595 al. 2}{\cciv} prévoit que le bail
										ne lui sera opposable que pour la période de 9 ans en cours
										au jour de la fin de l’usufruit.

									\item Lorsque le bail est d’une durée au plus égale à 9 ans,
										ce bail est opposable au \NP sans accord préalable
										sauf\footnote{Ces cas s'apparentent à des cas de fraude.} :
										\begin{itemize}
											\item bail rural conclu ou renouvelé plus de 3 ans avant
												expiration du bail courant ;

											\item bail d’habitation conclu ou renouvelé plus de 2 ans
												avant expiration du bail courant.
										\end{itemize}
								\end{itemize}
							\end{enumerate}

						\item Lorsque l'on a une indivision, les indivisaires sont en
							principes soumis à la règle de l'unanimité.
						Il y a des exceptions, et l'\articleDu{815-3}{\cciv} prévoit
							notamment qu'un ou plusieurs indivisaires titulaires d'au moins
							deux tiers des droits indivis peuven accomplir des actes
							d'administration sans avoir besoin de l'accord des autres
							indivisaires.
						Cette règle ne s'impose qu'en abscence de mandat, qui peut être
							tacite\footnote{La \JP a reconnu qu'un indivisaire qui a agi au vu
							et au su des autres indivisaires ai reçu de manière tacite un
							mandat.}
						Concernant le \textbf{bailleur indivisaire},
							\begin{itemize}
								\item \textbf{A défaut de convention d’indivision} en
									conjonction des \articlesDuEtSuivants{1873-1}{\cciv} et de
									l'\articleDu{815-3}{\cciv}, une décision à la majorité des
									\sfrac{2}{3} des droits indivis est possible s'il s'agit d'un acte d’administration.

								Cependant les autres indivisaires doivent être informés,
									à peine d’inopposabilité du bail.

								\item \textbf{Si acte de disposition} Il y a nécessité d'une
									décision unanime, la théorie du mandat tacite ne pouvant
									trouver alors à s'appliquer.
								Il est bien sûr possible de disposer d'un mandant spécial
									(écrit).

								L'\articleDu{815-5-1}{\cciv} permet à un, ou plusieurs,
									indivisaire qui détient \sfrac{2}{3} des droits indivis de
									conclure un contrat relevant des actes de dispositions, à
									condition du respect d'une procédure requérant une convention
									notarié et une homologuation par le juge.
								En pratique, de part sa lourdeur, cette procédure est peu
									utilisée, d'autant qu'il existe la possibilité de saisir le juge directement en cas de refus ou de silence des autres indivisaires (Cf. \ref{cciv:815-4et5}).

								\item \textbf{En présence d’un indivisaire hors d’état de manifester sa volonté}\label{cciv:815-4et5} une habilitation judiciaire possible aux termes des \articlesCodifies{815-4} et \mbox{815-5} du \cciv.
							\end{itemize}

							\smallbreak Un bail consenti par un indivisaire sans l’accord des autres est inopposable et non pas nul\footnote{C'est le cas d'un bail de la chose d’autrui}.

							\smallbreak L’indivision bailleresse. Une indivision n'a pas de personnalité juridique. Le bail conclu au nom d’une indivision, dépourvue de personnalité juridique, est nul de nullité absolue.

							\smallbreak Un co-indivisaire peut se faire consentir un bail sur la chose indivise --- à distinguer du droit de jouissance exclusif qui oblige au versement d’une indemnité à l’indivision\footnote{\ArticleDu{815-9 al. 2}{\cciv}}.

							Dans ce cas un écrit s'impose pour éviter toutes les difficultés que cette situation engendre en matière de preuve.

							\smallbreak Lorsque l'indivision est organisée, le gérant dispose des pouvoirs attribués à chaque époux sur les biens communs\footnote{\articleDu{1873-6}{\cciv}}.

							\jurisCourDeCas*[16-13063]{\civTrois}{16/03/2017} \hl{A }

						\item Lorsque le \textbf{bailleur est marié} : deux cas se présentent.
							\begin{itemize}
								\item Si l’immeuble est un bien peronnel, propre, en particulier
									dans le cadre de la séparation de bien, ou de la communauté
									réduite aux acquêts, le principe est celui de la liberté.
								Cependant, si le bien est affecté au logement de la
									famille\footnote{\ArticleDu{215 al. 3}{\cciv}}, il doit
									préalablement à la mise à bail requérir l'accord de son
									conjoint.
								A défaut, le bal est nul de nullité relative\footnote{C'est-à-
									dire qu'elle ne peut être invoquée que par le conjoint dont l'accord n'a pas été préalblement sollicité. Cette nullité se prescrit par les deux ans à compter du jour où le-dit conjoint à connaissance de cette situation.}.

								\item Si l’immeuble est un bien commun, il faut distinguer selon
									le type de bail : bail \enquote{acte d'administration} ou bail
									\enquote{acte de disposition}.
								Si le bail est de type \enquote{acte d'administration} l'\articleDu{1425}{\cciv}
								\begin{itemize}
									\item L'\articleDu{1425}{\cciv} requiert l'accord des deux époux pour baux ruraux et commerciaux. A défaut, l'acte est nul de nullité relative.

									\item art 595 CC pour les autres baux cf usufruitier
								\end{itemize}
							\end{itemize}
					\end{enumerate}

					\begin{focus}{Saisie immobiliere et bail}

						Principe art L 312 4 CPCE « Les baux consentis par le débiteur
						après l'acte de saisie sont, quelle que soit leur durée,
						inopposables au créancier poursuivant comme à l'acquéreur La
						preuve de l'antériorité du bail peut être faite par tout moyen »

						Civ 2 27 févr 2020 n 18 19174 « la délivrance d'un
						commandement valant saisie immobilière n'interdit pas la
						conclusion d'un bail ou la reconduction tacite d'un bail
						antérieurement conclu, et que le bail, même conclu après la
						publication d'un tel commandement est opposable à
						l'adjudicataire qui en a eu connaissance avant l'adjudication »

						cf art 1743 C civ

					\end{focus}

				\subparagraph{Pouvoirs du preneur}

					Acte d’administration

					co titularité art 1751 CC

					colocation qualité de locataire de chacun des co
					preneurs définition du bail loi ALUR du 24 mars 2014
					art 8 nouv Loi du 6 07 1989 « location d'un même
					logement par plusieurs locataires constituant leur
					résidence principale et formalisée par la conclusion d'un
					contrat unique ou de plusieurs contrats entre les
					locataires et le bailleur »

					\begin{focus}{Colocation}
						Solidarité des co locataires si clause en ce sens

						Limite loi 6 07 1989 en cas de congé libération immédiate si nouveau locataire ou dans les 6 mois en l’absence de nouveau locataire en presence d’une clause de solidarité
					\end{focus}

		\subsubsection{Le contenu du contrat de bail}

			\paragraph{Détermination du bien loué}

				\subparagraph{Bien objet du bail}

					Tout ou partie de la propriété d’un immeuble (emplacement de stationnement local couvert

					Limites :
					\begin{itemize}
						\item bien du domaine public

						\item droits réels ayant un caractère personnel (droit d’usage
						et d’habitation art. 631 CC)

						\item immeuble insalubres ou indécents , immmeubles divisés
						: L 111 6 1 CCH et s. + L 634 1 et s. CCH + L 1331 22
						et s. CSP
					\end{itemize}

					% Diapo n°42
					L’indivisibilité de la chose objet du bail cesse à son
					expiration

					Biens meubles objet du contrat de location
					immobilière indivisibilité
					% Fin de la diapo n°42

					\begin{focus}{Bien objet du bail division}

						Interdiction de divisions de locaux indécents cf L 111 6 1
						CCH

						Une autorisation de travaux en vue de créer plusieurs
						locaux d’habitation peut être imposée par décision de l’EPCI
						ou du CM dans les zones d’habitat dégradé contrôle des
						normes de décence cf L 111 6 1 1 CCH) et dans les zones
						U ou AU contrôle des proportions et tailles de logement cf
						PLU cf L 111 6 1 2 CCH)

						Autorisation de division procédure art. L 111 6 1 3 CCH :
						\begin{itemize}
							\item Demande suivant formes fixées par arrêté

							\item Décision notifiée dans les 15 jours de la réception de la
							demande Le défaut de réponse dans le délai de quinze jours
							vaut autorisation

							\item Le défaut d'autorisation de division est sans effet sur le bail

							\item Préfet peut ordonner le paiement d'une amende au plus
							égale à 15 000 En cas de nouveau manquement dans un
							délai de trois ans le montant maximal de cette amende est
							porté à 25 000 au profit de l’ANAH
						\end{itemize}
					\end{focus}

				\subparagraph{Déclaration de mise en location}

					Zones délimitées par l’EPCI ou le CM (art L 634 1 CCH)

					Délimitation des catégories et caractéristiques des logements
					concernés

					Déclaration dans les 15 jours suivant la conclusion du contrat
					de location (cf arrêté du 27 mars 2017

					Déclaration renouvelée à chaque nouvelle mise en location

					L'absence de déclaration est sans effet sur le bail

					Le bénéfice du paiement en tiers payant des aides
					personnelles au logement est subordonné à la production du
					récépissé de la déclaration de mise en location

					\begin{focus}{Sanction du non respect de l’obligation de déclaration}
						Amende au plus égale à 5 000 le produit en est
						intégralement versé à l'Agence nationale de l'habitat

						L'amende prononcée par le Préfet est proportionnée à la
						gravité des manquements constatés relatifs aux obligations
						de déclaration et ne peut être prononcée plus d'un an à
						compter de la constatation des manquements
					\end{focus}

				\subparagraph{Autorisation de mise en location} art. L635 1 et s. CCH

					L’EPCI ou la CM peut délimiter des zones soumises à
					autorisation préalable de mise en location sur les territoires
					présentant une proportion importante d'habitat dégradé
					suivant modalités fixées par délibération

					L’autorisation peut être refusée ou soumise à conditions
					lorsque le logement est susceptible de porter atteinte à la
					sécurité des occupants et à la salubrité publique
					(prescription de travaux

					Délai d’instruction 1 mois Le défaut de réponse
					vaut autorisation préalable de mise en location

					L'autorisation préalable de mise en location doit être
					renouvelée à chaque nouvelle mise en location

					Cette autorisation doit être jointe au contrat de bail
					à chaque nouvelle mise en location ou relocation

					\begin{focus}{Sanction}

						Amende au plus égale à 5 000 En cas de
						nouveau manquement dans un délai de trois ans le
						montant maximal de cette amende est porté à 15
						000

						L’amende prononcé par le Préfet profite à l’ANAH

						L'amende est proportionnée à la gravité des
						manquements constatés et ne peut être prononcée
						plus d'un an à compter de la constatation des
						manquements
					\end{focus}

			\paragraph{Fixation du loyer}

				Prix du bail à distinguer des charges accessoires

				Prix déterminé ou déterminable et sérieux (non vil

				Loyer en nature possible limites

				Encadrement des loyers pour les locaux à usage
				d’habitation (L 6 07 1989 art 17 à 18 Art 18

				Indexation (L 112 1 et s. CMF)

			\paragraph{Durée du bail}

				Principe liberté bail à durée déterminée ou
				indéterminée

				Limite art 1709 CC prohibition des baux
				perpétuels ( NA)(cf art 1210 CC)

				Reconduction tacite ou renouvellement poursuite
				du bail qui devient à durée indéterminée sauf
				statut particulier cf art 1214 et 1215 CC) à
				distinguer de la prorogation cf art 1213 CC)


	\subsection{Exécution du contrat de bail}

		Articles 1719 et s. CC

		Règles supplétives tant droits et obligations du
		preneur que du bailleur

		\subsubsection{Droits et obligations du preneur}

		article 1728 CC
		\begin{enumerate}
			\item Droits et obligations inhérentes à l’usage du bien
			\begin{itemize}
				\item Droit d’user de la chose louée raisonnablement

				\item Obligation d’utiliser le bien conformément à sa
				destination
			\end{itemize}

			\item Obligations financières

			\item Obligation aux réparations locatives et obligation de
			restitution
		\end{enumerate}

		\paragraph{Droits et obligations inhérentes à l’usage du bien}

			\subparagraph{Jouissance raisonnable}

				Abus de jouissance fait d’user de la chose louée
				dans des conditions anormales ou excessives
				emportant une dépréciation de la chose ou une gêne
				pour le bailleur ou les tiers

				Le preneur est responsable des pertes et
				dégradations survenant pendant la durée du bail (art
				1732 CCC) de son fait et de celui des « personne de sa
				maison ou de ses sous locataires » (art 1735 CC)

				Présence d’animaux article 10 loi n 70 598 du
				9 07 1970

			\subparagraph{Respect de la destination du bien loué}

				Pas de changement de destination sans accord du bailleur

				Si destination mixte simple faculté

				Domiciliation des entreprises L 123 10 et L 123
				11 du C comm n’entraîne pas de changement
				d’affectation des lieux loués

		\paragraph{Obligations financières}

			\subparagraph{Loyer}
				Paiement des loyers

				\begin{itemize}
					\item Modalités ( époque, lieu)
						\begin{itemize}
							\item époque suivant les usages

							\item Lieu
								\begin{itemize}
									\item Principe art 1342 6 CC loyer quérable

									\item Si loyer stipulé portable le bailleur n’a pas à en réclamer le paiement
								\end{itemize}
						\end{itemize}

					\item Rétention du loyer cause de résiliation du bail sauf impossibilité totale
					d’utiliser les lieux (exception d’inexécution
				\end{itemize}

				Retard générateur d’intérêts cf art 1344 1 CC

			\subparagraph{Rétention des loyers jurisprudence}

				Le locataire qui suspend le paiement de ses loyers sans,
				préalablement, demander en justice l’autorisation de les
				consigner ne peut pas opposer au bailleur l’exception
				d’inexécution, même si celui ci ne remplit pas ses
				obligations

				Le locataire encourt la résiliation de son bail s'il retient le
				loyer Cass 3 e civ 13 7 2010 n 09 67999

				Seule exception : impossibilité absolue d'utiliser les lieux
				(Cass. 3e civ. 5 10 2017 n 16 19614)

			\subparagraph{Charges}
				Détermination conventionnelle sauf règles impératives

				Modalités cf loyer

				Régularisation suivant prévision du bail sauf bail Loi de
				1989

		\paragraph{Garanties des obligations du preneur}

			Dépôt de garantie (limitation Loi de 1989 Débiteur de la
			restitution

			Privilège mobilier art 2332 1 CC sur les meubles
			garnissant les lieux loués même s’ils n’appartiennent pas
			au locataire rendu efficace par l’obligation de
			garnissement cf art 1752 CC

			Cautionnement non étendu aux obligations resultant de la
			prorogation du bail (art 1740 CC)

			Assurance

			\begin{focus}{Portée de la clause de solidarité}
				Civ . 3 ème , 12 janv. 2017 n 16 10324

				Tous les copreneurs solidaires sont tenus au
				paiement des loyers et des charges jusqu’à
				l’extinction du bail, quelle que soit leur
				situation personnelle

				La solidarité ne s’étend pas aux indemnités
				d’occupation, faute de stipulation expresse en
				ce sens
			\end{focus}

		\paragraph{Obligation aux réparations locatives}

			En cours de bail et jusqu’à la remise des clés entretenir la chose louée et
			effectuer les réparations locatives qui s’imposent (art 1720 1754 1755 CC
			pour les baux L 89 liste limitative D n 87 712 du 26 08 1987

			Cas particulier des travaux d’économie d’énergie (L 1989 art 23 1
			participation maximale imputable au locataire 50 du montant des travaux
			bénéficiant au preneur et justifiés Décret n 2009 1439 du 23 11 2009

			Exonération de l’obligation pour les réparations liés à la vétusté ou la force
			majeure (art 1755 CC)

			Sanction en fin de bail sauf si inexécution des travaux de nature à nuire à la
			chose louée résiliation

		\paragraph{Obligation de restitution}

			Remise des clés

			Lieux libérés de tous les meubles appartenant au locataire
			\begin{itemize}
				\item Remise en état initial des lieux loués (art 555 CC inapplicable
				constructions ou plantations sur le terrain du bailleur Civ
				3 1 er juin 2010 n 08 21254

				\item Vétusté à apprécier

				\item Incidence de l’absence d’état des lieux (art 1731 CC
				présomption de remise d’un bien en bon état de réparation
			\end{itemize}

		\subsubsection{Droits et obligations du bailleur}

			\begin{enumerate}
				\item Obligation de délivrance
				\item Obligation d’entretien
				\item Obligation de garantie
			\end{enumerate}

			\paragraph{Obligation de délivrance}

				\subparagraph{Obligation d’information}

					Obligation à contenu variable suivant le bien objet du bail
					\begin{itemize}
						\item Bail soumis à la loi du 6/07/1989 : art. 3 1 et 3 3 de la loi
							\begin{itemize}
								\item surface habitable

								\item DDT : ERNMT, DPE, CREP, état amiante (cf. décret ) et état des installations
								intérieures de gaz et d’électricité

								\item Notice d’information Arrêté du 29 mai 2015

								\item Un document comportant l’indication claire et précise des zones de bruit définies
								par un plan d’exposition au bruit des aérodromes prévu à l’article L. 112 6 du
								code de l’urbanisme (C. u rb a rt L. 112 11 au 1 er juin 2020)
							\end{itemize}

						\item Bail à usage d’habitation : DDT, logement décent

						\item Locaux à usage professionnel de plus de 2000 m2 : annexe environnementale (L
						125 9 1 CE) + amiante

						\item Locaux à usage professionnel de moins de 2000 m2 : DPE + ERNMT + amiante
					\end{itemize}

				\subparagraph{Obligation de délivrance d’une chose conforme aux prévisions des parties}

					Délivrance conforme aux prévisions des parties local
					principal et accessoires

					Délivrance suivant les délais convenus

					Logement décent (art 1719 CC)

					Délivrance en bon état de réparation de toutes
					espèces (art 1720 CC) pas de distinction entre les
					réparations locatives et les grosses réparations

			\paragraph{Obligation d’entretien}
				Toutes réparations autres que locatives (grosses réparations
				vétusté

				Limite : art. 1722 CC : destruction du bien
				\begin{itemize}
					\item destruction totale par cas fortuit bail résilié de plein droit
					sans dédommagement du preneur Si faute de l’une des
					parties résiliation du bail cf 1741 CC dommages intérêts

					\item destruction partielle option du preneur résiliation du bail
					ou diminution du loyer
				\end{itemize}

				Limites conventionnelles texte supplétif

			\paragraph{Obligation de garantie}

				Garantie des vices et défauts art 1721 CC
				\begin{itemize}
					\item objet tous vices ou défauts empêchant l’usage de la chose
					louée interprétation large dès que le vice occasionne un
					inconvénient sérieux au preneur garantie qui concerne tant
					la chose louée que ses accessoires

					\item sanction résiliation du bail ou réduction du loyer
					dommages intérêts
				\end{itemize}

				Garantie d’éviction du fait personnel du bailleur (art 1719 CC)
				troubles de droit comme de fait

				Garantie d’éviction du fait des tiers (art 1727 CC) troubles de
				droit

			\begin{focus}{Faculté de remplacement}
				Nouvel article 1222 du code civil

				Après mise en demeure, le créancier peut, dans un délai et à
				un coût raisonnables, faire exécuter lui même l'obligation ou,
				sur autorisation préalable du juge, détruire ce qui a été fait
				en violation de celle ci Il peut demander au débiteur le
				remboursement des sommes engagées à cette fin

				Il peut aussi demander en justice que le débiteur avance les
				sommes nécessaires à cette exécution ou à cette
				destruction
			\end{focus}

			\begin{focus}{Inexécution et réduction du prix}
				Art 1217 nouveau le créancier insatisfait peut « obtenir une réduction du prix »

				Art 1223 nouveau al 1 « En cas d'exécution imparfaite de la prestation, le créancier peut, après mise en demeure et s'il n'a pas encore payé tout ou partie de la prestation, notifier dans les meilleurs délais au débiteur sa décision d'en réduire de manière proportionnelle le prix
				 L'acceptation par le débiteur de la décision de réduction de prix du créancier doit être rédigée par écrit »

				Art 1223 nouveau al 2 « Si le créancier a déjà payé, à défaut d'accord
				entre les parties, il peut demander au juge la réduction de prix »
			\end{focus}


	\subsection{Incidents contractuels}

		\begin{enumerate}
			\item Extinction du bail art 1741 résiliation arrivée du
				terme (art 1741 perte de la chose, confusion des
				qualités de preneur et de propriétaire

			\item Décès d’une partie

			\item Vente du bien loué

			\item Cession de bail
		\end{enumerate}

		\subsubsection{Extinction du bail}

			\paragraph{Perte de la chose louée}

				Art. 1722 C. civ.

				En cas de perte totale, le bail est résilié de plein droit à
				la demande de l’une ou l’autre des parties sans que le
				juge ait de pouvoir d’appréciation

				En cas de perte partielle, seul le locataire, qui bénéficie
				d’une option, peut demander la résiliation Cass 3 e civ
				1 2 1995 n 92 21 376

				Question comment caractériser la perte totale

				Cass civ 3 ème 8 mars 2018 n 17 11439
				Article 1722 C civ doit être assimilée à la
				destruction en totalité de la chose louée
				l’impossibilité absolue et définitive d’en user
				conformément à sa destination ou la nécessité
				d’effectuer des travaux dont le coût excède sa
				valeur
				=)
				permet au bailleur d’échapper à l’obligation de
				remise en état

		\subsubsection{Décès d’une partie}

			Principe absence d’incidence (art 1742 CC)
			transmission aux héritiers

			En droit commun, le décès du preneur entraine la ... intuitu personae

			Dans les baux spécifique il est possible que la loi ne permette pas la transmission

			Exceptions convention contraire, renonciation du
			bénéficiaire de la transmission, bail loi de 1948 pour les
			descendants majeurs du locataire

			\paragraph{Transmission du bail}

				Portée de la règle toute transmission à titre gratuit ou onéreux y
				compris vente par adjudication
				–
				Date certaine nécessaire A défaut connaissance du bail par acquéreur
				ou acceptation de poursuivre en connaissance de cause
				–
				Acquéreur subrogé sans rétroactivité dans les droits et obligations du
				vendeur à l’égard du locataire et de sa caution sauf clause contraire
				(Ass Plén 6 dec 2004
				–
				Sort du dépôt de garantie cas particulier L 1989
				–
				Clause dérogatoire valable sauf règles d’\OP

		\subsubsection{La vente du bien loué}

			lorsque le baileur . le \cciv nous indique que cela n'a pas de perturbaton, le contrat s'impose au nouveau popriétaire à la condition que le bail soit authentique ou de date certaine.

			La \JP est allée au-delà dès lors que l'acquéreur avait connaissance ...

			Le contrat de bail ets ainsi couramment joint à l'acte de vente ...

			Le nouveau propriétaire devient tenu des toutes les nouvelles à compter du transfert de pro. Ce ne sont que les fautes commises à compter du ...

			\paragraph{Vente du bien loué L. 1989}

				Limitation du droit pour l’acquéreur d’un bien acquis loué de
				donner congé Limitation variable suivant motif de congé
				\begin{itemize}
					\item Congé pour vente trois ans au moins après la date
					d’acquisition

					\item Congé pour reprise 2 ans au moins après l’acquisition
				\end{itemize}


			\paragraph{Droit de préemption du locataire}

				Droit de préemption conventionnel pacte de préférence

				Droit de préemption légal.
				\begin{itemize}
					\item vente après division
					\item vente en bloc
					\item congé donné pour vendre
				\end{itemize}

				\subparagraph{Droit de préemption art. 10 L. 31 déc. 1975}

					\begin{itemize}
						\item Locaux concernés locaux à usage d’habitation ou à usage
						mixte effectivement occupés par le locataire à usage
						d’habitation quel que soit le bail conclu local principal et
						locaux accessoires si vente du local principal et acccessoire
						\item Locaux vendus en bloc ou séparément (à distinguer d’une
						vente d’un bâtiment dans son entier ou de la totalité des locaux
						à usage d’habitation ou mixte
						\item locaux issus d’une première division d’un immeuble entier ou
						d’une subdivision
					\end{itemize}

					Nature des droits cédés pleine propriété (Quid si
					démembrement de propriété ou droits sociaux d’une société
					d’attribution

					Opérations concernées première vente consécutive à la
					division de l’immeuble par lots ou à la subdivision de tout ou
					partie de l’immeuble par lots (à distinguer d’une simple
					division parcellaire

					Si 1\iere{} vente anéantie (résolution ou annulation) considérée
					comme jamais intervenue

				\subparagraph{Cas d’exclusion du droit de préemption du locataire}

					Vente à un parent ou allié jusqu’au 4 inclus interprétation stricte
					exclusion du cas d’une vente à société d’un seul associé parent ou allié
					jusqu’au 4 degré

					Vente de l’immeuble dans son entier ou ensemble des lots à usage
					d’habitation ou mixte si première vente de l’immeuble entier puis deuxième
					vente après division droit de préemption

					Exclusion des opérations de partage et opérations ne constituant pas une
					vente au sens strict (ex mutation ATG, apport en société

				\subparagraph{Titulaires du droit de préemption}

					Chaque co preneurs et co titulaire du bail (art 1751 CC)
					ou occupants de bonne foi

					Le droit de préemption ne profit qu’au titulaire qui occupe
					effectivement les lieux loués exclut les personnes
					morales et les locataires ayant consenti une sous location
					totale

					Droit à préemption individuel chaque titulaire doit être
					traité comme s’il était seul la renonciation de l’un est
					inopposable aux autres notification à chacun

				\subparagraph{Mise en oeuvre du droit de préemption}

					Notification d’une offre au locataire (LRAR, acte d’huissier ou ou
					la remise en main propre contre récépissé ou émargement ( CP C ,
					art. 667)667)(sanction : NR de la

					Contenu :
					\begin{itemize}
						\item indication du prix (hors commission d’agence Civ 3 17 déc
						2008 n 07 15943 et des conditions de la vente tous
						éléments pouvant déterminer le consentement du locataire

						\item reproduction des 5 premiers alinéa de l’article 10 L 1975 si
						pluralité de locataires § 1 et 2 art 4 D du 30 06 1977
					\end{itemize}

				\subparagraph{Notification de l’offre au locataire}

					Auteur propriétaire cas particulier des indivisions et propriété
					démembrée à considérer ou mandataire disposant d’un mandat
					spécial acte de disposition)

					Destinataire chaque titulaire du bail ou occupants de bonne foi Si
					locataire sous tutelle notification à son représentant légal (D
					1977 art 6 Notification opposable au conjoint si son existence n’a
					pas été préalablement portée à la connaissance du bailleur

					Durée de validité 2 mois à compter de sa réception délai
					décompté pour chaque locataire délai préfix

				\subparagraph{Forme et effets de l’acceptation de l’offre}

					Forme LRAR ou forme de portée équivalente

					Délai 2 mois de la réception de la notification

					Destinataire personne émettrice de l’offre

					Contenu acceptation sans équivoque et sans réserve (à défaut
					refus possibilité d’indiquer que l’acceptation est subordonnée à
					l’obtention d’un prêt

					Conséquences réalisation de la vente dans les 2 ou 4 mois si prêt) A
					défaut nullité de l’acceptation du locataire

				\subparagraph{Forme et effets de la renonciation à l’offre}

					Forme manifestation exprès de volonté ou écoulement du
					délai de 2 mois sans manifestation de volonté ou de 2 4
					mois sans réalisation de la vente Ne prive pas le locataire
					de son droit au bail

					Effets la loi ALUR du 24 mars 2014 information de la
					commune du lieu de situation du bien titulaire d’un droit de
					prioritée disposition déclarée inconsitutionnelle au
					11 01 C Const 9 janv 2018 n 2017 683 QPC

				\subparagraph{En l’absence de préemption}

					Le vendeur peut vendre son bien occupé à la personne de
					son choix aux conditions indiquées dans la notification
					faite au locataire

					Si vente à des conditions plus avantageuses nouvelle
					purge du droit de préemption selon les mêmes règles mais
					avec un délai d’acceptation réduit à 1 mois

					Sanction du non respect de la procédure nullité de la
					vente

				\subparagraph{Droit de préemption du locataire en cas de vente par adjudication art. 10 II Loi 1975}

					Information du locataire et convocation à l’adjudication cf formes
					supra, 1 mois au moins avant la date de l’adjudication (D n 77
					742 du 30 06 1977 art 7

					Notification du jugement d’adjudication entre le 10 et le 15 jour
					suivant l’adjudication surenchère possible)

					Sanction à défaut de convocation régulière le locataire dispose
					d’un droit de substitution dans le mois de la date à laquelle il a eu
					connaissance de l’adjudication sauf si adjudication au profit d’un
					indivisaire

				\subparagraph{Droit de préemption du locataire et vente en bloc art. 10 1 L. 1975}

					Introduit par la loi Aurillac du 13 juin 2006

					Opérations concernées vente d’un immeuble à usage d’habitation ou
					mixte en totalité et en une seule fois et vente de la totalité des parts ou
					actions d’une société d’attribution détenant un tel immeuble

					Conditions cumulatives de l’ouverture du droit de préemption du locataire :
					\begin{itemize}
						\item immeuble comportant plus de 5 logements Loi ALUR)

						\item acquéreur ne s’engageant pas à proroger les baux en cours pour 6 ans à compter de l’acte authentique de vente (engagement à insérer dans l’acte à peine de nullité de la vente
					\end{itemize}

				\subparagraph{Exceptions au droit de préemption du locataire art. 10 1 L. 1975}

					Préalablement à la vente le bailleur communique au maire de la
					commune du lieu de situation de l’immeuble le prix et les
					conditions de vente si DPU la DIA remplace cette notification)
					Permet à la commune de mettre en oeuvre son droit de
					préemption de l’article L 210 2 C urb

					Notification au locataire cf conditions de forme de l’art 10 L
					1975 (prix du logement concerné et prix de vente global
					reproduction des termes de l’article 10 1 I A L 1975 à peine de
					nullité documents DT et projet de règlement de copropriété
					si plus de 10 logements informations imposées par accords de
					1998 2005

				\subparagraph{Suites de la notification}

					Offre de vente valable 4 mois à compter de sa reception

					Silence gardé par le locataire refus

					Si acceptation vente doit être réalisée dans les 2 mois de
					l’acceptation ou dans les 4 mois si recours à un prêt L’immeuble
					est alors soumis au régime de la copropriété

					Si refus vente ne peut se faire à des conditions plus avantageuses
					sauf purge d’un nouveau droit de préemption délai réduit à un
					mois pour l’acceptation

				\subparagraph{Accords collectifs}

					2 accords collectifs se sont succédés 9 juin 1998 et 16
					mars 2005

					applicables en cas de mise en vente d’un immeuble de
					plus de 10 logements avec ou sans congé lorsque le
					bailleur est une personne morale cf art 41 ter L
					23 12 1986 sauf SCI entre parents et alliés jusqu’au 4
					degré inclus

					obligation d’information spécifique des locataires
					(sanction nullité

				\subparagraph{Congé donné pour vendre droit de préemption}

					loi du 6 juillet 1989

					Art . 15 / art. 11 1
					=) renvoi

		\subsubsection{Cession de bail}

			\paragraph{Cession du droit au bail}

				Article 1717 Code civil principe d’autorisation sauf
				convention contraire ou règle d’ordre public contraire (cf
				sous location)

				Sanction résiliation du bail et inopposabilité de la
				convention conclu en violation de la clause du bail

			\paragraph{Effet de la cession de bail}

				\subparagraph{Rapports avec le bailleur}
					transmission intégrale au cessionnaire
					des droits et obligations en résultant sans novation de débiteur
					le preneur primitif reste tenu des obligations du bail vis à vis du
					bailleur
					En
					cas de cessions successives le bailleur peut agir contre l’un
					quelconque des cessionnaires Si clause de solidarité s’impose
					aux cessionnaires successifs

				\subparagraph{Rapports cédant -- cessionnaire}
					le cessionnaire dispose d’une
					action directe contre le bailleur pour obtenir l’exécution de ses
					obligations nées du bail bailleur peut opposer les exceptions
					qu’il aurait pu opposer au cédant

			\paragraph{Droit commun des contrats}

				Cession de la qualité de partie au contrat

				Article 1216 al 3 CC «La cession doit être constatée par
				écrit, à peine de nullité »

				Article 1216 1 CC «Si le cédé y a expressément consenti, la
				cession de contrat libère le cédant pour l'avenir

				A défaut, et sauf clause contraire, le cédant est tenu
				solidairement à l'exécution du contrat »

				Article 1216 2 CC « Le cessionnaire peut opposer au cédé les
				exceptions inhérentes à la dette, telles que la nullité, l'exception
				d'inexécution, la résolution ou la compensation de dettes connexes Il
				ne peut lui opposer les exceptions personnelles au cédant Le cédé
				peut opposer au cessionnaire toutes les exceptions qu'il aurait pu
				opposer au cédant

				Article 1216 3 CC « Si le cédant n'est pas libéré par le cédé, les
				sûretés qui ont pu être consenties subsistent Dans le cas contraire,
				les sûretés consenties par le cédant ou par les tiers ne subsistent
				qu'avec leur accord

				Si le cédant est libéré, ses codébiteurs solidaires restent tenus
				déduction faite de sa part dans la dette »

\section{Le droit spécial des baux civils}

	Réglementations d’ordre public

	Baux soumis à la loi du 6 juillet 1989 dite loi Mermaz
	Baux non soumis à la loi du 6 juillet 1989 dite loi
	Mermaz

	\subsection{Champ d’application de la loi du 6 juillet 1989}

		\subsubsection{Les locations exclues de la loi du 6 juillet 1989}

			Locations totalement exclues
			\begin{enumerate}
				\item locaux à usage professionnel
				\item locaux à usage rural
				\item logements foyers
				\item logements de fonction
				\item logements meublés touristiques
				\item contrat de cohabitation intergénérationnelle solidaire
			\end{enumerate}


			Locations partiellement exclues
			\begin{itemize}
				\item logements meublés à usage de résidence principale
				\item b ail mobilité
				\item ogements du secteur HLM et logements conventionnés
				\item logements régis par la loi du 1/09/1948
			\end{itemize}


		\subsubsection{Les locations incluses dans le champ d’application de la loi du 6 juillet 1989}

			Locaux compris dans un immeuble construit avant le 1 er sept
			1948 à usage d’habitation ou professionnel ainsi que les
			locaux accessoires à ces locaux

			Les occupants de ces logements preneur ou occupant de
			bonne foi bénéficient du régime jusqu’à la vacance des
			locaux Exclusion des locataires entrés dans les lieux après le
			23 déc 1986

			Depuis la loi ENL de 2006 les descendants des occupants ne
			bénéficient plus du régime de faveur (sauf vie effective depuis
			un an art 5 L 1948

	\subsection{Régime de la loi du 6 juillet 1989}

		Droit au maintien dans les lieux sauf occupation insuffisante
		moins de 8 mois au cours d’une année de location)

		Congé pour certains travaux ou pour habiter avec obligation
		de relogement ( sauf preneur protégé

		Fixation autoritaire du prix des loyers suivant un prix au m 2
		fixé par voie réglementaire au 1 er juillet de chaque année
		suivant classement lié aux équipements et au confort

		\subsubsection{Formation du contrat}

		\subsubsection{Obligations des parties en cours de bail}

\pagebreak

\pdfbookmark[0]{Table des matières}{toc}\tableofcontents

\end{document}
